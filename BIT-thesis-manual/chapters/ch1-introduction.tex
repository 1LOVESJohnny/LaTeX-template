%%==================================================
%% ch1.tex for BIT Master Thesis
%% modified by yang yating
%% version: 0.2
%% last update: Feb 16th, 2017
%%==================================================

\chapter{使用简介}
\label{chap:what}

这是北京理工大学硕士(博士)学位学位论文~\LaTeX~ 模板。目前网上有两个版本的北理工~\LaTeX~ 模板“2012大眼小蚂蚁版”和“2016汪卫版”,均以上海交通大学的模为基础。本模板在此两个模板的基础上依据《北京理工大学博士、硕士学位论文撰写规范》修改,进一步完善成熟,使得模板能够被北理工硕士博士广泛使用。

\section{模板说明}
\label{sec:features}

这个模板的中文解决方案是~\XeTeX/\LaTeX~ 。参考
文献建议使用~BibTeX~管理,可以生成符合国标~GBT7714~风格的参考文献列表。模
板在~Windows~和~Linux~下测试通过,更详细的系统要求请参考
\ref{sec:requirements}。

模板的外观表现和功能都放
在~bitmaster-xetex.cls~和~bitmaster-xetex.cfg~中,在对外观进行细微调整
时,只需要更新这两个文件,不需要对.tex源文件做修改。这也给模板更新带来了
极大方便。

该模板的主要功能说明如下:

\begin{itemize}
\item \inv 依据《北京理工大学博士、硕士学位论文撰写规范》修改,按照本文档说明使用,即可撰写符合《北京理工大学博士、硕士学位论文撰写规范》的学位论文;
\item \inv 使用~\XeTeX~ 引擎处理中文;
\item \inv 包含中文字符的源文件(.tex, .bib, .cfg),编码都使用UTF-8;
\item \inv 使用~BibTeX~管理参考文献。参考文献表现形式(格式)受~.bst~控制,方便
  在不同风格间切换,目前生成的列表符合国标GBT7714要求;
\item \inv 可以直接插入EPS/PDF/JPG/PNG格式的图像。
\item \inv 硕士模板的格式受~bitmater-xetex.cls~和~bitmaster-xetex.cfg~控制,博士模板的格式受~bitdoctor-xetex.cls~和~bitdoctor-xetex.cfg~控制方便模
  板更新和模板修改。
\end{itemize}

\section{安装配置环境}
\label{sec:requirements}

~XeLaTeX~ 对应的XeTeX对中文字体的支持更好,允许用户使用操作系统字体来代替TeX的标准字体。所以本模板使用~XeLaTeX~ 引擎处理中文。要使用这个模板协助你完成研究生学位论文的创作,下面的条件必须满足:

\begin{itemize}
\item \inv 操作系统字体目录中有中文字体(adobe或windows字体均可);
\item \inv \TeX~系统有~\XeTeX~引擎;
\item \inv \TeX~系统有~ctex~宏包;
\item \inv \TeX~系统的fontspec宏包(fontspec.sty)足够新;
\item \inv 你有使用~\LaTeX~的经验。
\end{itemize}

为了安装便捷,推荐安装Ctex套装。

Ctex套装下载地址 (http://www.ctex.org/CTeXDownload)


安装完成后运行Winedit,便可对tex进行编写。也可使用texmaker等其他的编辑器进行编写。

\section{快速使用}
\label{sec:process}

安装完ctex套装后,一般而言所需的环境就配置好了。
模板使用~\XeTeX~ 引擎提供的~xelatex~的命令处理,作用于“主控文档”demo.tex。
并且,可以省略扩展名。完整的处理流程是:

{\color{blue}
\begin{enumerate}
\item[] ~\verb|xelatex -no-pdf -{}-interaction=nonstopmode demo|
\item[] ~\verb|bibtex demo| 
\item[] ~\verb|xelatex -no-pdf -{}-interaction=nonstopmode demo|
\item[] ~\verb|xelatex -{}-interaction=nonstopmode demo|
\end{enumerate}}

运行bibtex的时候会提示一些错误,可能是~{{\sc Bib}\TeX}~对UTF-8支持不充
分,一般不影响最终结果。留意因为拼写错误导致的``找不到文献错误''即可。

加入~\verb|--interaction=nonstopmode|~参数是不让错误打断编译过程。
\XeTeX~ 仍存在一些宏包兼容性问题,所以会产生一些莫名其妙的错误(通常是重定
义错误),而这些错误通常不会影响最终的编译结果。
  
为方便使用,处理过程已经写入BIT-thesis-run.sh(for Linux)和BIT-thesis-run.bat(for Windows)批处理文件中。编写完修改完tex文件后,直接运行对应.sh或者.bat文件即可。


